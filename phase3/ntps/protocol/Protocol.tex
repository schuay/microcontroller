\documentclass[12pt,a4paper,titlepage,oneside]{article}
\usepackage{graphicx}            % fuer Bilder
\usepackage{epsfig}              % fuer EPS Bilder
\usepackage{listings}            % fuer Programmlistings
%\usepackage{german}              % fuer deutsche Umbrueche
\usepackage[latin1]{inputenc}    % fuer Umlaute
\usepackage{amssymb,amsmath,amsthm}
\usepackage[usenames,dvipsnames]{color}
\usepackage[pdfborder={0 0 0}]{hyperref}

\definecolor{Brown}{cmyk}{0,0.81,1,0.60}
\definecolor{OliveGreen}{cmyk}{0.64,0,0.95,0.40}
\definecolor{CadetBlue}{cmyk}{0.62,0.57,0.23,0}
\definecolor{gray}{gray}{0.5}

\lstset{
    language=C,                             % Code langugage
    basicstyle=\ttfamily,                   % Code font, Examples: \footnotesize, \ttfamily
    keywordstyle=\color{OliveGreen},        % Keywords font ('*' = uppercase)
    commentstyle=\color{gray},              % Comments font
    captionpos=b,                           % Caption-position = bottom
    breaklines=true,                        % Automatic line breaking?
    breakatwhitespace=false,                % Automatic breaks only at whitespace?
    showspaces=false,                       % Dont make spaces visible
    showtabs=false,                         % Dont make tabs visible
    morekeywords={__attribute__},           % Specific keywords
}

%***************************************************************************
% note: the template is in English, but you can use German for your
% protocol as well; in that case, remove the comment from the
% \usepackage{german} line above
%***************************************************************************


%***************************************************************************
% enter your data into the following fields
%***************************************************************************
\newcommand{\Vorname}{Jakob}
\newcommand{\Nachname}{Gruber}
\newcommand{\MatrNr}{0203440}
\newcommand{\Email}{e0203440@student.tuwien.ac.at}
\newcommand{\Part}{I}
\newcommand{\Tutor}{Patrik Fimml}
%***************************************************************************


%***************************************************************************
% generating the document from Protocol.tex:
%       "latex Protocol"        generates a .dvi file
%       "latex Protocol"        repeat to get correct table of contents
%       "xdvi Protocol &"       shows the .dvi file on viewer
%       "dvips Protocol.dvi -o Protocol.ps"      generates a postscript file
%
%***************************************************************************

%---------------------------------------------------------------------------
% include all the stuff that is the same for all protocols and students
\input ProtocolHeader.tex
%---------------------------------------------------------------------------

\begin{document}

%---------------------------------------------------------------------------
% create titlepage and table of contents
\MakeTitleAndTOC
%---------------------------------------------------------------------------


%***************************************************************************
% This is where your protocol starts
%***************************************************************************


%***************************************************************************
\section{Overview}
%***************************************************************************

\subsection{Quickstart}


%---------------------------------------------------------------------------
\subsection{Connections,  External Pullups/Pulldowns}
%---------------------------------------------------------------------------

\bConnections{What}{}
SW15, MMC Dipswitches & Disabled \\
PB4 & Ethernet RST \\
PB0 & Ethernet CS \\
PD2 & Ethernet INT \\
PB1 & Ethernet SCK \\
PB3 & Ethernet MISO \\
PB2 & Ethernet MOSI \\
PB4 & Ethernet RST \\
SW14, RTC Dipswitches (5, 7) & Enabled \\
SW13, RS232-A Dipswitches (5, 6) & Enabled \\
SW13, RS232-B TX Dipswitch (8) & Enabled \\
SW13, Touchscreen Dipswitches (1-4) & Enabled \\
SW15, Backlight Dipswitch (8) & Enabled \\
J18 & Set to VCC \\
J15 & Disconnected \\
\eConnections

All other dipswitches in SW12-SW15 should be disabled.

%---------------------------------------------------------------------------
\subsection{Design Decisions}
%---------------------------------------------------------------------------

\begin{itemize}

 \item
 
\end{itemize}


%---------------------------------------------------------------------------
\subsection{Specialities}
%---------------------------------------------------------------------------

\begin{itemize}

 \item
 
\end{itemize}


%***************************************************************************
\section{Main Application}
%***************************************************************************


%***************************************************************************
\section{Problems}
%***************************************************************************

Apart from the usual early adopter issues (incomplete specification, bugs in
used tools, toolchain setup issues), my main problems were the initial lack of
debugging outputs (since one UART was used by the application, and the other
could not be used due to pin conflicts), and an obscure compiler bug which took
several days to track down (thanks to Andreas Hagmann for the help).

%***************************************************************************
\section{Work}
%***************************************************************************

\begin{tabular}{|l|c|}
\hline
reading manuals, datasheets	& 1 h	\\
program design			& 1 h	\\
programming			& 13 h	\\
debugging			& 25 h	\\
questions, protocol		& 5 h	\\
\hline
{\bf Total}			& 45 h	\\
\hline
\end{tabular}

%***************************************************************************
\section{Theory Tasks}
%***************************************************************************

\subsection{GPS Fault Tolerance}

To tolerate $f > 0$ faulty modules of a total of $n > 2 \cdot f + 1 >
0$ GPS modules, the f-tolerant mean $avg_f(t_1, ... , t_n)$ is used, which is
defined as follows:

Of the ordered set $t_{(1)}, ... , t_{(n)}$, the $f$ largest and $f$ smallest
$t_i$ are discarded, and the average of the remaining values $t_{(f + 1)}, ...,
t_{(n - f)}$ is used as the result.

GPS accuracy is restricted such that the difference between the actual and the
measured GPS time is always below a threshold $\pi$.

\subsubsection{Fault bounds of a single controller}

Assuming we have f faulty modules, there are at most f modules for which
$|t-gps_f(t)| \leq \pi$ does not hold. By applying $avg_f(.)$, the f largest and
smallest values are discarded, and from the threshold guarantee of non-faulty
models it follows that the remaining $n - 2 \cdot f$ values are within the
interval $t - \pi \leq gps_i(t) \leq t + \pi$.

Finally, from $x_{(1)} \leq avg(x_{(1)}, ..., x_{(n)}) \leq x_{(n)}$, it follows
that $|t - avg_f(gps_1(t), ..., gps_n(t))| \leq \pi$ for all $t$.

\subsubsection{Fault bounds of two controllers}

The difference between the measurement of two controllers can be shown easily
using the results we have just acquired. We have shown that one controller using
$avg_f(.)$ with f faulty GPS modules will set its time to within $[t - \pi, t +
\pi]$. The maximum possible difference between the measurements of two
controllers is achieved if one measures the minimum $t - \pi$, and the other the
maximum $t + \pi$, with the difference between the two being $2 \cdot \pi$.

Thus, the maximum difference between the fault-tolerant measurement of two
controllers $d \leq 2 \cdot \pi$.

\subsection{Rounding of Values}
\subsection{Rounding of Larger Values}


%***************************************************************************
\newpage
\appendix
\section{Listings}
\small{
%***************************************************************************

\subsection{main.c}
\lstinputlisting{../src/main.c}

\subsection{UdpConfig.h}
\lstinputlisting{../src/UdpConfig.h}

\subsection{NtpsAppC.nc}
\lstinputlisting{../src/NtpsAppC.nc}

\subsection{Rtc.nc}
\lstinputlisting{../src/Rtc.nc}

\subsection{GpsTimerParser.nc}
\lstinputlisting{../src/GpsTimerParser.nc}

\subsection{HplDS1307C.nc}
\lstinputlisting{../src/HplDS1307C.nc}

\subsection{Enc28j60C.nc}
\lstinputlisting{../src/Enc28j60C.nc}

\subsection{PingP.nc}
\lstinputlisting{../src/PingP.nc}

\subsection{NtpsC.nc}
\lstinputlisting{../src/NtpsC.nc}

\subsection{GpsTimerParserC.nc}
\lstinputlisting{../src/GpsTimerParserC.nc}

\subsection{UdpTransceiverC.nc}
\lstinputlisting{../src/UdpTransceiverC.nc}

\subsection{UserInterface.nc}
\lstinputlisting{../src/UserInterface.nc}

\subsection{PingC.nc}
\lstinputlisting{../src/PingC.nc}

\subsection{UdpTransceiverP.nc}
\lstinputlisting{../src/UdpTransceiverP.nc}

\subsection{GpsTimerParser.h}
\lstinputlisting{../src/GpsTimerParser.h}

\subsection{TimeC.nc}
\lstinputlisting{../src/TimeC.nc}

\subsection{minunit.h}
\lstinputlisting{../src/minunit.h}

\subsection{Time.nc}
\lstinputlisting{../src/Time.nc}

\subsection{DS1307C.nc}
\lstinputlisting{../src/DS1307C.nc}

\subsection{Rtc.h}
\lstinputlisting{../src/Rtc.h}

\subsection{UserInterfaceC.nc}
\lstinputlisting{../src/UserInterfaceC.nc}

\subsection{HplDS1307.nc}
\lstinputlisting{../src/HplDS1307.nc}

\subsection{HplDS1307.h}
\lstinputlisting{../src/HplDS1307.h}

%***************************************************************************
}% small
\end{document}
