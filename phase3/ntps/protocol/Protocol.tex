\documentclass[12pt,a4paper,titlepage,oneside]{article}
\usepackage{graphicx}            % fuer Bilder
\usepackage{epsfig}              % fuer EPS Bilder
\usepackage{listings}            % fuer Programmlistings
%\usepackage{german}              % fuer deutsche Umbrueche
\usepackage[latin1]{inputenc}    % fuer Umlaute
\usepackage{amssymb,amsmath,amsthm}
\usepackage[usenames,dvipsnames]{color}
\usepackage[pdfborder={0 0 0}]{hyperref}

\definecolor{Brown}{cmyk}{0,0.81,1,0.60}
\definecolor{OliveGreen}{cmyk}{0.64,0,0.95,0.40}
\definecolor{CadetBlue}{cmyk}{0.62,0.57,0.23,0}
\definecolor{gray}{gray}{0.5}

\lstset{
    language=C,                             % Code langugage
    basicstyle=\ttfamily,                   % Code font, Examples: \footnotesize, \ttfamily
    keywordstyle=\color{OliveGreen},        % Keywords font ('*' = uppercase)
    commentstyle=\color{gray},              % Comments font
    captionpos=b,                           % Caption-position = bottom
    breaklines=true,                        % Automatic line breaking?
    breakatwhitespace=false,                % Automatic breaks only at whitespace?
    showspaces=false,                       % Dont make spaces visible
    showtabs=false,                         % Dont make tabs visible
    morekeywords={__attribute__},           % Specific keywords
}

%***************************************************************************
% note: the template is in English, but you can use German for your
% protocol as well; in that case, remove the comment from the
% \usepackage{german} line above
%***************************************************************************


%***************************************************************************
% enter your data into the following fields
%***************************************************************************
\newcommand{\Vorname}{Jakob}
\newcommand{\Nachname}{Gruber}
\newcommand{\MatrNr}{0203440}
\newcommand{\Email}{e0203440@student.tuwien.ac.at}
\newcommand{\Part}{I}
\newcommand{\Tutor}{Patrik Fimml}
%***************************************************************************


%***************************************************************************
% generating the document from Protocol.tex:
%       "latex Protocol"        generates a .dvi file
%       "latex Protocol"        repeat to get correct table of contents
%       "xdvi Protocol &"       shows the .dvi file on viewer
%       "dvips Protocol.dvi -o Protocol.ps"      generates a postscript file
%
%***************************************************************************

%---------------------------------------------------------------------------
% include all the stuff that is the same for all protocols and students
\input ProtocolHeader.tex
%---------------------------------------------------------------------------

\begin{document}

%---------------------------------------------------------------------------
% create titlepage and table of contents
\MakeTitleAndTOC
%---------------------------------------------------------------------------


%***************************************************************************
% This is where your protocol starts
%***************************************************************************


%***************************************************************************
\section{Overview}
%***************************************************************************

\subsection{Quickstart}

Connect the ethernet module and RS-232A to enable gpsfeed+ and network
communication. Turn on the board. The current RTC time will be displayed on the
GLCD. When a GPS time event arrives, it is also added to the GLCD display.
Pressing the touchscreen buttons will set the RTC \emph{on arrival of the next
GPS time event}. The board will respond to pings, NTP requests, and UDP packets
on unused ports.

%---------------------------------------------------------------------------
\subsection{Connections,  External Pullups/Pulldowns}
%---------------------------------------------------------------------------

\bConnections{What}{}
SW15, MMC Dipswitches & Disabled \\
PB4 & Ethernet RST \\
PB0 & Ethernet CS \\
PD2 & Ethernet INT \\
PB1 & Ethernet SCK \\
PB3 & Ethernet MISO \\
PB2 & Ethernet MOSI \\
PB4 & Ethernet RST \\
SW14, RTC Dipswitches (5, 7) & Enabled \\
SW13, RS232-A Dipswitches (5, 6) & Enabled \\
SW13, RS232-B TX Dipswitch (8) & Enabled \\
SW13, Touchscreen Dipswitches (1-4) & Enabled \\
SW15, Backlight Dipswitch (8) & Enabled \\
J18 & Set to VCC \\
J15 & Disconnected \\
\eConnections

All other dipswitches in SW12-SW15 should be disabled.

%---------------------------------------------------------------------------
\subsection{Design Decisions}
%---------------------------------------------------------------------------

\begin{itemize}

\item Unit testing
    Many important functions can be (with a little care) written to make unit
    testing possible. Tests are important for improving application stability
    and allow for higher confidence when debugging and altering the code. These
    tests are executed by running \emph{make check} in the project root.

\item Use the same underlying data type for rtc\_time\_t and timedate\_t.
    This simplifies code which handles interactions between the RTC and GPS
    modules.

\item Time-critical work is performed directly in events instead of tasks.
    For example, I2C packet read handling is not extracted to a task since we
    don't want to miss any read events. Another alternative would have been to
    use a buffer area for overflow storage, but within the scope of this
    assignment it seemed like overkill.

\item Use only RTC bulk read/write methods
    The HplDS1307 interface offers ways to perform targeted read/write
    operations of specific register locations. However, most (if not all) of
    the implemented functionality in DS1307C requires reading or writing
    several different register sections. Therefore, it seemed to be a better
    solution (avoiding function overhead, complicated code, additional
    potential error sources) to use bulk read/writes exclusively.

\item The RTC is set only on incoming time event
    Instead of setting the RTC immediately, we wait for the next incoming GPS
    time event to ensure accuracy.

\item The RTC is polled once per second

\item The touchscreen is polled once every 50 milliseconds
    50 ms should be long enough to avoid wasting too many resources, and short
    enough not to miss any touch events. The touchscreen needs to be polled
    because it doesn't differentiate between idle and touched states.

\item Do not store performance-critical constants in PROGMEM
    This includes, for example, the \emph{days until month} table in TimeC.
    Time calculations need to be fast, so we'd rather lose a few bytes of RAM
    than performance.

\item Do not store GLCD string constants in PROGMEM
    The GLCD functions do not accept PROGMEM strings.

\item Treat GPS time as UTC
    gpsfeed+ sends the local time in its GPS sentences. This causes NTP replies
    to have a two hour offset to the actual time (+1 timezone, +1 summer time).
    Since the GPS signal should send UTC, we ignore this issue, blaming it on
    gpsfeed+ upstream.
 
\end{itemize}


%---------------------------------------------------------------------------
\subsection{Specialities}
%---------------------------------------------------------------------------

\begin{itemize}

\item Unit testing

\item Partial compilation
    The application can be compiled without extras (Touchscreen, GLCD,
    Ethernet) by adding \emph{-DNOEXTRAS} to CFLAGS.

\item Robust and failure-tolerant GPRMC sentence parser
    Every complete GPRMC sentence should be parsed successfully, even if the
    rest of the data stream is invalid.
 
\end{itemize}


%***************************************************************************
\section{Modules}
%***************************************************************************

%***************************************************************************
\subsection{Main Application}
%***************************************************************************

The main application has very little to do. All components are initialized, and
the user interface, RTC, GPS and ethernet modules are glued together by
reacting to touch events, incoming time information and NTP requests.

The RTC is polled once per second. Upon receival of updated RTC or GPS times,
they are displayed on the GLCD. When a button touch event is received, the RTC
is updated whenever the next incoming time event occurs.

If an NTP request is received, the reply packet is returned using the latest
cached RTC time.

%***************************************************************************
\subsection{User Interface}
%***************************************************************************

UserInterfaceC handles interaction with the GLCD and touchscreen. It is
responsible for drawing the UI and signalling button press events to the
application. The touchscreen is polled every 50 ms - if the returned
coordinates are within the button areas, a button press is emitted.

%***************************************************************************
\subsection{Time Conversion Utilities}
%***************************************************************************

TimeC encapsulates a couple of time utility functions, including day of week
determination and conversion between NTP timestamps and rtc\_time\_t structs.
Two-way conversion makes tasks like adding offsets to a specific timedate very
easy.

%***************************************************************************
\subsection{ICMP and UDP Handlers}
%***************************************************************************

ICMP echo (Ping) and destination unreachable packets are handled in the PingP
and UdpTransceiverP modules. The only notable things to watch out for are that
reply packets must contain the correct type and code and that destination
unreachable packets contain a portion the original request packet to help the
sender identify which packet is referenced by the reply.

%***************************************************************************
\subsection{Unit Tests}
%***************************************************************************

Unit testing is performed using minunit
(\url{http://www.jera.com/techinfo/jtns/jtn002.html}).

Important functions are adapted for testing using conditional compilation.

%***************************************************************************
\subsection{GPS Parser}
%***************************************************************************

The GpsTimerParserC module parses an incoming UART GPS data stream for time
data, and emits a signal when a time has been received. Time information is
contained in GPRMC sentences and have the following format:

\$GPRMC,123519,A,4807.038,N,01131.000,E,022.4,084.4,230394,003.1,W*6A

Fields are separated by commas; the time is in field 0 (123519 represents
12:35:19), and the date in field 8 (230394 represents March 23, 1994).

A state machine is used to parse the incoming data stream. It is designed to
tolerate transmission failures, incorrect and unexpected data. Every complete
GPRMC sentence should be parsed successfully, even if the rest of the data
stream is invalid.

The parser is also fairly space-efficient, requiring only a 6 byte buffer plus
one byte to store the current index.

Upon completion of a successful parse, all interested parties are notified by
emitting the newTimeDate signal.

%***************************************************************************
\subsection{Real Time Clock}
%***************************************************************************

The Real Time Clock (RTC) module uses a layered approach: the interface
HplDS1307 (module DS1307C)is responsible for low level tasks such as reading
and writing to individual RTC registers, while interface Rtc (module DS1307C)
encapsulates common functions such as starting, stopping, setting, and querying
the RTC.

Communication with the RTC is performed using the I2C protocol. Access to the
bus is protected by using the Resource interface. Most of the tasks performed
by HplDS1307C are split-phase and consist of multiple steps, therefore a state
machine is used to control execution flow.

DS1307C uses HplDS1307C to implement common functions. Again, a state machine
is used to handle control flow in split-phase commands. A few private utility
functions to handle tasks such as converting between BCD and normal integer
representation are also included.


%***************************************************************************
\section{API Documentation}
%***************************************************************************

API functions are documented in Doxygen style within the source code.

%***************************************************************************
\section{Problems}
%***************************************************************************

Apart from the usual early adopter issues (incomplete specification, bugs in
used tools, toolchain setup issues), my main problems were the initial lack of
debugging outputs (since one UART was used by the application, and the other
could not be used due to pin conflicts), and an obscure compiler bug which took
several days to track down (thanks to Andreas Hagmann for the help).

%***************************************************************************
\section{Work}
%***************************************************************************

\begin{tabular}{|l|c|}
\hline
reading manuals, datasheets	& 1 h	\\
program design			& 1 h	\\
programming			& 13 h	\\
debugging			& 25 h	\\
questions, protocol		& 10 h	\\
\hline
{\bf Total}			& 50 h	\\
\hline
\end{tabular}

%***************************************************************************
\section{Theory Tasks}
%***************************************************************************

\subsection{GPS Fault Tolerance}

To tolerate $f > 0$ faulty modules of a total of $n > 2 \cdot f + 1 >
0$ GPS modules, the f-tolerant mean $avg_f(t_1, ... , t_n)$ is used, which is
defined as follows:

Of the ordered set $t_{(1)}, ... , t_{(n)}$, the $f$ largest and $f$ smallest
$t_i$ are discarded, and the average of the remaining values $t_{(f + 1)}, ...,
t_{(n - f)}$ is used as the result.

GPS accuracy is restricted such that the difference between the actual and the
measured GPS time is always below a threshold $\pi$.

\subsubsection{Fault bounds of a single controller}

Assuming we have f faulty modules, there are at most f modules for which
$|t-gps_f(t)| \leq \pi$ does not hold. By applying $avg_f(.)$, the f largest and
smallest values are discarded, and from the threshold guarantee of non-faulty
models it follows that the remaining $n - 2 \cdot f$ values are within the
interval $t - \pi \leq gps_i(t) \leq t + \pi$.

Finally, from $x_{(1)} \leq avg(x_{(1)}, ..., x_{(n)}) \leq x_{(n)}$, it follows
that $|t - avg_f(gps_1(t), ..., gps_n(t))| \leq \pi$ for all $t$.

\subsubsection{Fault bounds of two controllers}

The difference between the measurement of two controllers can be shown easily
using the results we have just acquired. We have shown that one controller using
$avg_f(.)$ with f faulty GPS modules will set its time to within $[t - \pi, t +
\pi]$. The maximum possible difference between the measurements of two
controllers is achieved if one measures the minimum $t - \pi$, and the other the
maximum $t + \pi$, with the difference between the two being $2 \cdot \pi$.

Thus, the maximum difference between the fault-tolerant measurement of two
controllers $d \leq 2 \cdot \pi$.

\subsection{Rounding of Values, Rounding of Larger Values}

The solution to the assembler rounding assignments is presented as commented
source code.

The functions main\_div\_truncate, main\_div\_round\_up,
main\_div\_round\_to\_nearest correspond to assignments 2a, 2b, and 2c.

main\_big\_div\_truncate, main\_big\_div\_round\_up,
main\_big\_div\_round\_to\_nearest correspond to assignments 3a, 3b, and 3c.

\lstinputlisting[language={[x86masm]Assembler}]{round.s}

%***************************************************************************
\newpage
\appendix
\section{Listings}
\small{
%***************************************************************************

\subsection{main.c}
\lstinputlisting{../src/main.c}

\subsection{UdpConfig.h}
\lstinputlisting{../src/UdpConfig.h}

\subsection{NtpsAppC.nc}
\lstinputlisting{../src/NtpsAppC.nc}

\subsection{Rtc.nc}
\lstinputlisting{../src/Rtc.nc}

\subsection{GpsTimerParser.nc}
\lstinputlisting{../src/GpsTimerParser.nc}

\subsection{HplDS1307C.nc}
\lstinputlisting{../src/HplDS1307C.nc}

\subsection{Enc28j60C.nc}
\lstinputlisting{../src/Enc28j60C.nc}

\subsection{PingP.nc}
\lstinputlisting{../src/PingP.nc}

\subsection{NtpsC.nc}
\lstinputlisting{../src/NtpsC.nc}

\subsection{GpsTimerParserC.nc}
\lstinputlisting{../src/GpsTimerParserC.nc}

\subsection{UdpTransceiverC.nc}
\lstinputlisting{../src/UdpTransceiverC.nc}

\subsection{UserInterface.nc}
\lstinputlisting{../src/UserInterface.nc}

\subsection{PingC.nc}
\lstinputlisting{../src/PingC.nc}

\subsection{UdpTransceiverP.nc}
\lstinputlisting{../src/UdpTransceiverP.nc}

\subsection{GpsTimerParser.h}
\lstinputlisting{../src/GpsTimerParser.h}

\subsection{TimeC.nc}
\lstinputlisting{../src/TimeC.nc}

\subsection{minunit.h}
\lstinputlisting{../src/minunit.h}

\subsection{Time.nc}
\lstinputlisting{../src/Time.nc}

\subsection{DS1307C.nc}
\lstinputlisting{../src/DS1307C.nc}

\subsection{Rtc.h}
\lstinputlisting{../src/Rtc.h}

\subsection{UserInterfaceC.nc}
\lstinputlisting{../src/UserInterfaceC.nc}

\subsection{HplDS1307.nc}
\lstinputlisting{../src/HplDS1307.nc}

\subsection{HplDS1307.h}
\lstinputlisting{../src/HplDS1307.h}

%***************************************************************************
}% small
\end{document}
