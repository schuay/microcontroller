%***************************************************************************
% MCLab Protocol Template
%
% Embedded Computing Systems Group
% Institute of Computer Engineering
% TU Vienna
%
%---------------------------------------------------------------------------
% Vers.	Author	Date	Changes
% 1.0	bw	10.3.06	first version
% 1.1	bw	25.4.06	listing is in a different directory
% 1.2	bw	24.5.06	tutor has to be listed on title page
% 1.3	bw	16.6.06	statement about no plagiarism on title page (sign it!)
%---------------------------------------------------------------------------
% Author names:
%       bw      Bettina Weiss
%***************************************************************************

\documentclass[12pt,a4paper,titlepage,oneside]{article}
\usepackage{graphicx}            % fuer Bilder
\usepackage{epsfig}              % fuer EPS Bilder
\usepackage{listings}            % fuer Programmlistings
%\usepackage{german}              % fuer deutsche Umbrueche
\usepackage[latin1]{inputenc}    % fuer Umlaute
\usepackage{amssymb,amsmath,amsthm}
\usepackage[usenames,dvipsnames]{color}

\definecolor{Brown}{cmyk}{0,0.81,1,0.60}
\definecolor{OliveGreen}{cmyk}{0.64,0,0.95,0.40}
\definecolor{CadetBlue}{cmyk}{0.62,0.57,0.23,0}
\definecolor{gray}{gray}{0.5}

\lstset{
    language=C,                             % Code langugage
    basicstyle=\ttfamily,                   % Code font, Examples: \footnotesize, \ttfamily
    keywordstyle=\color{OliveGreen},        % Keywords font ('*' = uppercase)
    commentstyle=\color{gray},              % Comments font
    captionpos=b,                           % Caption-position = bottom
    breaklines=true,                        % Automatic line breaking?
    breakatwhitespace=false,                % Automatic breaks only at whitespace?
    showspaces=false,                       % Dont make spaces visible
    showtabs=false,                         % Dont make tabs visible
    morekeywords={__attribute__},           % Specific keywords
}

%***************************************************************************
% note: the template is in English, but you can use German for your
% protocol as well; in that case, remove the comment from the
% \usepackage{german} line above
%***************************************************************************


%***************************************************************************
% enter your data into the following fields
%***************************************************************************
\newcommand{\Vorname}{Jakob}
\newcommand{\Nachname}{Gruber}
\newcommand{\MatrNr}{0203440}
\newcommand{\Email}{e0203440@student.tuwien.ac.at}
\newcommand{\Part}{I}
\newcommand{\Tutor}{Patrik Fimml}
%***************************************************************************


%***************************************************************************
% generating the document from Protocol.tex:
%       "latex Protocol"        generates a .dvi file
%       "latex Protocol"        repeat to get correct table of contents
%       "xdvi Protocol &"       shows the .dvi file on viewer
%       "dvips Protocol.dvi -o Protocol.ps"      generates a postscript file
%
%***************************************************************************

%---------------------------------------------------------------------------
% include all the stuff that is the same for all protocols and students
\input ProtocolHeader.tex
%---------------------------------------------------------------------------

\begin{document}

%---------------------------------------------------------------------------
% create titlepage and table of contents
\MakeTitleAndTOC
%---------------------------------------------------------------------------


%***************************************************************************
% This is where your protocol starts
%***************************************************************************


%***************************************************************************
\section{Overview}
%***************************************************************************

\subsection{Quickstart}

To get started, set up connections as described in the next section,
insert the correct wiimote MAC addresses in src/wiimote.h, and install
using 'make install'. 

Turn on the board. Press the sync buttons on both wiimotes to connect them.
Once both have been connected, the game starts. Control the paddles using
the up and down buttons.

%---------------------------------------------------------------------------
\subsection{Connections,  External Pullups/Pulldowns}
%---------------------------------------------------------------------------

\bConnections{What}{}
J15 & Connected to PF0 \\
J18 & Connected to VCC \\
SW13 & RS-232 A switches enabled \\
SW15 & GLCD/LCD backlight enabled \\
\eConnections

All other peripherals are connected as described in the specification.

%---------------------------------------------------------------------------
\subsection{Design Decisions}
%---------------------------------------------------------------------------

\begin{itemize}


 \item Performance considerations in the WT41 HAL module
 
 This module requires high performance from the receive routines. Therefore,
 implement ISR's directly in the module to avoid function pointer indirection.
 Reenable interrupts in ISR's as early and often as possible. Focus on avoiding
 overhead in the ringbuffer implementation instead of safety and modularity.
 Let one 'thread' to process the ringbuffer contents while simultaneously
 allowing new data to be written into the ringbuffer.
 
 The transmit portions are much less critical.
 
 
 \item Performance considerations in the ringbuffer implementation
 
 At first, the ringbuffer was implemented as a separate module. When this
 proved to be too slow, it was reimplemented as a part of the WT41 HAL module.
 
 To discern empty and full states, we never fill the entire buffer. Therefore,
 \_rb\_write == \_rb\_read is an empty buffer, and \_rb\_wrte == \_rb\_read - 1 a full buffer.
 
 Read and write markers are integers marking the position in the array of the 
 next read and write.
 
 The ringbuffer size is a $2^n$ to allow for very efficient bit tricks for manipulating
 ringbuffer markers.

 
 \item GLCD drawing routines
 
 The high level drawing routines use the standard Gresenham algorithms for line and circle
 drawing. These avoid floating point math and are therefore very efficient for
 microcontroller usage. Rectangles are drawn using the line drawing function.
 
 
 \item Abstracting over GLCD chips
 
 The next read / write (x, y) coordinate is stored within the GLCD HAL module and kept
 up to date by the read and write functions. The correct chip and position is used
 automatically. When end of line is reached, the next position is (0, y+1). When end of line
 of the last line is reached, the next position is (0, 0).
 
 For simplicity, always communicate with either chip 0 or chip 1, never both. This way,
 we can use the same simple function interface for all communications.
 
 
 \item Ensuring GLCD and LCD timings
 
 Exact timings are achieved by writing the section in C, disassembling it, and inserting \_NOP()
 calls as required. Calculate a single cycle as 62.5 ns.
 
 Timings in the GLCD datasheet are more or less exact, while LCD timings sometimes seem to have
 no relation to reality.
 
 
 \item Text display
 
 Text display has been implemented using the LCD, simply because that part was available for home
 development and testing. 
 
 
 \item Game mechanics
 
 A very simple model has been chosen for ball movement: when hitting a surface, the ball
 is reflected with the same out- as incoming angle. If the paddle is hit, add/subtract a little
 from y-velocity depending on the hit location.
 
 Paddle collision detection is also very basic. At every processing step, we know the current and next
 $(x, y)$ coordinates. If either $x < 0$ (left of the table) or $x > \mathit{table_width}$ (right of the
 table), check whether a part of the paddle is located between $y_{cur}$ and $y_{next}$. 
 If yes, then a collision has occurred.
 
 
 \item Game control
 
 The game is controlled by pressing the up and down buttons on the wiimote. This was chosen
 over accelerometer control, because \textellipsis well, because motion control sucks. Keep it simple!
 
 
 \item Interface design
 
 We stick to a very basic interface. There are only a couple of things we want to know:
 Which controllers are connected? What's the score? Where are the paddles? Where is the ball?
 
 The former are displayed on the LCD, and the latter on the GLCD. 
 
 The GLCD displays only the paddles and the ball. Less drawing means better performance.
 
 
 \item Generic setup routines
 
 Generic setup routines are available for some hardware modules such as timers, UART,
 and the ADC. These help encapsulate low level access and keep accesses from other modules clean,
 simple, and easy to read. Options are available as needed, but most of them have been
 left out for simplicity.
 
 
 \item Debug output
 
 Debugging is absolutely necessary when programming. We can't use a debugger (since we don't
 have access to one), but we can at least enable debugging by printf's. To do this, set up a simple
 UART stream for stdout and stderr. Busy waiting is used for UART transmissions, but the functionality
 can be disabled entirely using compile time flags.
 
 
 \item Timer calculation
 
 Again, keep it simple by using a fixed prescaler factor of 1024, and only allow a limited range
 of time settings.
 
 
 \item MP3 playback
 
 Playback is only started when a point has been scored. There is no game in progress while playing the MP3,
 so we can transmit data as fast as we can until the entire sound has been played without slowing
 down progress. In fact, it turns out that the MP3 delay feels just right between scoring and resetting
 the board.
 
 
 \item Scheduler
 
 The scheduler consists of a timer interrupt triggered every 5 ms. It sets the RunLogic flag
 and increments ticks. Some tasks are only performed every n ticks, which is achieved by
 checking for $ticks mod n = 0$.
 
 
 \item Error recovery
 
 The only errors we can actually recover from are disconnections of wiimotes during the game.
 If that occurs, playback is paused, and only resumed once all wiimotes are connected.
 
 All other errors, such as disconnection of the bluetooth module, can cause the program to abort.
 
 
 \item ADC
 
 Every 50 ticks (250 ms), an ADC conversion is started. There is no point to check for volume changes
 more often. Also, changes in value less then ADC\_SMOOTHING are ignored to avoid constant volume changes
 because of cheap hardware.
 
 
 \item Wiimote connection handling
 
 Only a single wiimote connection can be established at a time. Handle connections by using the 
 wii\_connection\_change callback. On every timeout of wiiUserConnect, the callback is executed.
 On every (dis)connection, the callback is executed. So, all we need to do is check during the callback
 if any wiimotes are still disconnected, and if yes, try to connect to the next one.
 
 If all are connected, the game is set to GameRunning. Otherwise, the game is set to GamePaused. If
 the current state is PointScored, defer all state changes.
 
 
\end{itemize}


%---------------------------------------------------------------------------
\subsection{Specialities}
%---------------------------------------------------------------------------

\begin{itemize}
 \item  The visual representation of the board is very basic and could be much better looking. After spending
many hours on the drivers and other internals during vacation without tutor help, I didn't have much
energy remaining for extras.

 \item The driver implementation seems to be fairly clean and readable.
 
 \item Debugging is simplified by using a UART stream. 
 
 \item The UART bluetooth implementation is fast, yet simple.
 
\end{itemize}



%---------------------------------------------------------------------------
\subsection{API Documentation}
%---------------------------------------------------------------------------

The API is documented using Doxygen, and can be generated by running make doc.
It is then located in doc/html and doc/latex.


%***************************************************************************
\section{Main Application}
%***************************************************************************

The main application is controlled by a state machine consisting of the states 
GamePaused, GameRunning, and PointScored.


The startup state is GamePaused. In this state, nothing is drawn to the GLCD,
game logic is paused, and connections are attempted.
If both wiimotes are connected,
the state is changed to GameRunning.

In GameRunning, game logic is run, and the board is drawn to the LCD.
If a wiimote is disconnected, go back to GamePaused.
If a point is scored, enter PointScored.

In PointScored, an MP3 sound is sent to the MP3 module. No game logic
is run, nothing is drawn to the GLCD.
Once the MP3 is finished playing, either go back to GameRunning
if both wiimotes are still connected,
or to GamePaused if a wiimote has disconnected.

Once all initializations have been completed, the main loop is entered.
It consists of several background tasks which are executed depending on 
flags and the current state. A description of all tasks follows.

At least once every 5 ms, the logic tasks are enabled. These are executed only in 
the game running state and contain: input handling, game logic, and game rendering.
Of these, game logic is only executed every 20th tick (this is responsible for game
pace).

In the PointScored state, MP3 data transmission is handled in task\_mp3().

Every 50th tick, a new ADC conversion is started.

Finally, ADC results are processed whenever one is waiting.


%***************************************************************************
\section{Music Playback}
%***************************************************************************

%---------------------------------------------------------------------------
\subsection{SPI}
%---------------------------------------------------------------------------

The SPI implementation consists of the two required functions spiReceive,
spiSend, and initial setup in spi\_init. Busy waiting is used as allowed
by the specification.

%---------------------------------------------------------------------------
\subsection{Playback}
%---------------------------------------------------------------------------

Whenever a point is scored, a single MP3 sound is sent to the MP3 module.

This is achieved by continually reading a block from the SDCard and sending
it to the MP3 module until mp3Busy() returns true. When the MP3 module
requests more data the MP3DataRequested flag is set, and during the next main
loop iteration data transfer to the MP3 module continues.

Once the entire sound has been transferred, the state machine is switched back
to the GameRunning state.

The MP3DataRequested is ignored while the state machine is not in the
PointScored state.

This implementation avoids all complexities of handling MP3 data transfer
while other background tasks are running by dedicating the state PointScored
entirely to MP3 playback. It also feels natural to freeze the display for
a few moments after scoring points, so this works out very well.

%***************************************************************************
\section{WT41 HAL}
%***************************************************************************

The WT41HAL module handles low level communications with the bluetooth module.

Since transmission occurs at a very high rate and not keeping up with
transmitted packets results in communication failure, performance is the
main focus in this module.

During initialization, the bluetooth module is reset. This is handled by
Timer3, which is set to 5ms. UART3 is initialized once the bluetooth module
has been reset. Other initialization tasks are handled in the actual 
halWT41FcUartInit() function.

In addition to halWT41FcUartInit(), halWt41FcUartSend is provided for use by
libwii. 

Communication with the bluetooth module takes place as follows:

\begin{itemize}
 \item Transmit

 The next byte to be transmitted is kept in a buffer locally.
 
 Whenever the UART data register is empty, meaning we are ready to send a 
 byte of data to module, we send the buffered byte, disable the data register
 empty interrupt, and call the provided callback to let libwii know we are ready
 for more data. If RTS is set, use a pin change interrupt to wait for RTS to be
 unset before continuing the transmission.
 
 When libwii calls halWT41FcUartSend, the passed byte is buffered locally, and
 the data register empty interrupt is reenabled.
 
 \item Receive
 
 Performance is critical in this section. Therefore, the receive done interrupt
 service routine is implemented directly in the WT41 HAL module instead of relying
 on function pointer indirection as all others do. We also reenable interrupts
 as early and often as possible within this ISR.
 
 When a byte has been received, first check whether there are more than CTS\_HIGH
 byte free in our ringbuffer. If not, trigger flow control.
 
 Next, store the byte into the buffer. If we are already busy processing the bytes
 from the buffer, exit.
 
 Otherwise, take bytes from the buffer one by one, and call the provided libwii 
 callback. During this section, interrupts are enabled to allow new bytes to
 be put into the buffer.
 
 Once all bytes have been processed, exit.
 
\end{itemize}


%***************************************************************************
\section{LC-Display}
%***************************************************************************

%---------------------------------------------------------------------------
\subsection{GLCD}
%---------------------------------------------------------------------------

The GLCD implementation uses functions provided by the GLCD HAL to
draw on the GLCD. Lines, rectangles, and circles are all achieved using
Bresenham's algorithm without having to resort to floating point operations.

%---------------------------------------------------------------------------
\subsection{HAL GLCD}
%---------------------------------------------------------------------------

The GLCD HAL is responsible for low level tasks such as initializing the
GLCD chips, setting addresses, and sending/receiving commands and display
data. Busy waiting is used to ensure that the chips are ready to receive
commands as specified.

Communications with the chips are always either with chip 0 or chip 1,
not both. This results in a slower fill screen operation, but simplifies
the code.

%---------------------------------------------------------------------------
\subsection{LCD}
%---------------------------------------------------------------------------

The LCD HAL is responsible for low level tasks such as initializing the
LCD chips, setting addresses, and sending/receiving commands and display
data.




%***************************************************************************
\section{Problems}
%***************************************************************************

I encountered \emph{many} problems while working on avr-pong. Many of them
were due to bad / insufficient / incorrect documentation. Some of them
were caused by my own inexperience with microcontroller programming. Yet others
were bugs that could turn up in any programming project.

The first large hurdle was initializing the GLCD and displaying something on the
screen. I spent around half a day wondering why it wouldn't work until figuring
out that the RST pin was low active, and I'd been meticulously ensuring it 
was always 0.

Another issue with the GLCD was the undocumented necessity of double reads for
correct data. No thanks to the available datasheets.

The next issue I ran into was a (nonstatic) struct conflict the linker didn't warn
me about (even with -Wall -Wextra -pedantic). This resulted in a flag being set
in the main state struct which controlled when certain tasks would be executed.
Very hard to track down.

Bluetooth comms debugging also took a fair amount of time, especially due to having insufficient debug output. However, after optimizing the ringbuffer and interrupt, there was enough performance to avoid overruns.

The LCD timings again seemed to be very poorly and incorrectly documented. Occasionally, I had to use over 50x the specified amount. Also, the setup of 4 bit mode was not described sufficiently.

In general, programming it home and subsequently testing at the lab didn't work
out well. I always spent more time debugging and fixing things at the lab
than it would've taken to develop it all at the lab from the start.


%***************************************************************************
\section{Work}
%***************************************************************************

\begin{tabular}{|l|c|}
\hline
reading manuals, datasheets	& 2 h	\\
program design			& 2 h	\\
programming			& 20 h	\\
debugging			& 30 h	\\
questions, protocol		& 10 h	\\
\hline
{\bf Total}			& 64 h	\\
\hline
\end{tabular}



%***************************************************************************
\section{Theory Tasks}
%***************************************************************************


% Your answers should be brief but complete



\QuText{
\textbf{[2 Points] Buffersize and flow control:}
Assume the UART Setup as in the Application setup, that is, $\mathit{BAUD}$ bit/sec, 8N1. Furthermore, assume the ring buffer we have implemented is empty at the beginning, $\mathit{Buffersize}$ bytes large, and for simplicity forget about all the hardware and software implementation details and assume the UART Module could directly write in the buffer the flow control is triggered in the moment the last empty place in the buffer is written. An element is removed from the buffer in the moment the corresponding callback is called.

\begin{enumerate}

\item As in the application, the receive callback function is issued subsequently for every received character. Assume the bluetooth module sends datagramms of length $DG_{length}>\mathit{Buffersize}$. Derive a formula for the upper bound on the execution duration of the callback function $t_{CB}$ such that if one datagram is sent, the flow control is \textbf{not} triggered. The formula should depend on $\mathit{BAUD}$, $\mathit{Buffersize}$, and $DG_{length}$. Note that a UART-Mode of 8N1 is specified! Please be very careful when building the formula not to make an off-by-one mistake.
Additionally calculate the upper bound for the following values: $\mathit{BAUD}=10^{6}bit/sec$, $\mathit{Buffersize}=49Byte$, and $DG_{length}=64Byte$.

\item Assume the previously mentioned datagramms come periodically every $P_{DG}$ seconds. Derive a formula for the upper bound on the execution duration of the callback function $t_{CB}'$ such that no matter how long the system runs the flow control will \textbf{never} be triggered. Additionally, calculate the upper bound using the values from before and assume $P_{DG}=2.5*10^{-3}s$

\end{enumerate}
}

%***************************************************************************
% Theory task 1.1
%***************************************************************************

A 8N1 UART frame consists of 1 start bit, 8 data bits, and 1 stop bit = 10 bits.
Thus, a transmission of $DG_{length}$ uses exactly

\begin{displaymath}
    T_{length} = 10 * DG_{length}
\end{displaymath}

bits and takes

\begin{displaymath}
    T_{time} = T_{length} / \mathit{BAUD}
\end{displaymath}

seconds.
In that time, the callback function needs to process at least

\begin{displaymath}
    DG_{length} - \mathit{Buffersize} + 1
\end{displaymath}

bytes to avoid triggering the flow control. Therefore, the upper bound for the
callback function is

\begin{displaymath}
    t_{CB} = \frac{T_{time}}{DG_{length} - \mathit{Buffersize} + 1}
           = \frac{10 * DG_{length}}{\mathit{BAUD} * (DG_{length} - \mathit{Buffersize} + 1)}
\end{displaymath}


For $\mathit{BAUD}=10^{6}bit/sec$, $\mathit{Buffersize}=49Byte$, and
$DG_{length}=64Byte$, $t_{CB} = 4 * 10^{-5}$ seconds, or 40 $\mu$s.

%***************************************************************************
% Theory task 1.2
%***************************************************************************

Now, assuming that such transmission occur regularly every $P_{DG}$ seconds,
the callback needs to be short enough to handle a single transmission without
overflow, and process an entire transmission in $P_{DG}$ seconds to avoid long
term overflows.

To process an entire transmission in $P_{DG}$ seconds, a callback needs to
take at most

\begin{displaymath}
    \frac{P_{DG}}{DG_{length}}
\end{displaymath}

seconds.

This, to fulfill both conditions, the upper bound for a callback is

\begin{displaymath}
    t_{CB}' = min(t_{CB}, \frac{P_{DG}}{DG_{length}})
\end{displaymath}

For $P_{DG}=2.5*10^{-3}s$ and the other values as in task 1.2, $t_{CB}' = 39.0625 * 10^{-6}$ seconds.

\QuText{
\textbf{[1 Point] Pong vs.\ Billiard warmup:}
In the following assume for simplicity that the playground of Pong is
     a rectangle of height~1 and width~2 and that both players are
     perfect, that is, they always hit the ball with the tennis
     racket.
Also assume, that there are no pixels, that is, we are playing in a
     continuous environment.
Thus, this scenario is equivalent to a billiard board of height~1 and
     width~2 with one ball and no holes.
The lower left corned has Cartesian coordinates~$(0,0)$, the upper left
     corner~$(0,1)$ and the lower right corner~$(2,0)$.
Further assume (as in billiard) that the incidence angle is equal to
     the emergent angle whenever the ball hits a boundary or the
     players' tennis rackets.
In case the ball hits a corner, it thus simply reverts its direction
     at the corner.
The speed of the ball is constant during the whole game, i.e., we
     assume zero friction.

A game is called {\em periodic\/} if the ball repeatedly runs along
     the same finite track over and over again.

\medskip

For each natural number~$n$ give an initial position and initial
     direction of the ball such that it hits the border at least~$n$
     times before it reaches the same initial position with the same
     initial direction again (the game gets periodic). Prove that your solution is a majorizing series.
}

%***************************************************************************
% Theory task 2
%***************************************************************************

A path determined by an initial position and velocity can be visualized by imagining an
endless row of tables next to each other. The top and bottom borders reflect the ball
as expected, while left and right borders allow the ball to pass through.
When imagining all odd tables (table 1, 3, ...) as mirrored along the vertical axis,
this represents the full path.

We can further reduce this representation by fixing the initial position at $p
= (0,0)$, the x velocity $dx > 0$, and the y velocity $dy = 1$. The produced
path is a triangle wave with a period of $2 * dx$. Contacts with the bottom and
top borders occur every $dx$ units. The original position and velocity is
reached when a contact point occurs at $4 * n$.

Since we know the period of the wave $2 * dx$ and the position of returns to
the original position $4 * n$, it is now easy to determine the number of
traversed tables by using the least common multiple

\begin{displaymath}
    \mathit{traversed} = lcm(2 * dx, 4) = lcm(dx, 2)
\end{displaymath}

For simplicity, let's ignore contacts with upper and lower borders and only
count horizontal contacts.  Then, $\mathit{traversed} = \mathit{contacts}$. We
can use this as a lower bound for the number of border hits.

Now, if we assume $dx = n$, $\mathit{contacts} = lcm(n, 2) \geq n$, from which
follows:

For each natural number $n$, table borders will be hit at least $n$ times
before teaching the initial position and initial velocity again, if the initial
position is $p = (0, 0)$ and the initial velocity is $v = (n, 1)$.

\QuText{
\textbf{[2 Points] Pong vs.\ Billiard aperiodic gaming:}
Give an initial position and initial direction of the ball such that
     the ball will never reach the initial position again.
Formally prove that your solution is correct.
}

%***************************************************************************
% Theory task 3
%***************************************************************************

Again, let's assume the same representation of a ball path as above (an
infinite row of tables) and the same initial position $p = (0, 0)$.
The path is periodic iff $\mathit{traversed} = lcm(dx, 2)$ exists.

This is not the case if $dx$ is an irrational number.

Indirect proof. Assume that an irrational number exists such that $lcm(dx, 2)$
exists, or $dx * m = 2 * n$, with $m, n \in \mathbb{N}$.

\begin{displaymath}
    dx = \frac{2 * n}{m}
\end{displaymath}

RHS is a real number by the ratio $a/b$, where $a, b \in \mathbb{N}$. However,
dx is an irrational number, which by definition cannot be represented this way.

Therefore, it follows that for initial position $p = (0, 0)$ and velocity
$v = (dx, 1)$ with dx irrational, the path is nonperiodic.


%***************************************************************************
\newpage
\appendix
\section{Listings}
\small{
%***************************************************************************

Include EVERY source file of your Application (including headers)!!!
And EVERY file you have modified!

%---------------------------------------------------------------------------
\subsection{Application}
%---------------------------------------------------------------------------

\lstinputlisting{../src/main.c}

%---------------------------------------------------------------------------
\subsection{SPI}
%---------------------------------------------------------------------------

\lstinputlisting{../src/spi.c}
\lstinputlisting{../src/spi.h}

%---------------------------------------------------------------------------
\subsection{Timers}
%---------------------------------------------------------------------------

\lstinputlisting{../src/timer.c}
\lstinputlisting{../src/timer.h}

%---------------------------------------------------------------------------
\subsection{GLCD}
%---------------------------------------------------------------------------

\lstinputlisting{../src/glcd.c}
\lstinputlisting{../src/glcd.h}
\lstinputlisting{../src/glcd_hal.c}
\lstinputlisting{../src/glcd_hal.h}

%---------------------------------------------------------------------------
\subsection{LCD}
%---------------------------------------------------------------------------

\lstinputlisting{../src/lcd.c}
\lstinputlisting{../src/lcd.h}

%---------------------------------------------------------------------------
\subsection{ADC}
%---------------------------------------------------------------------------

\lstinputlisting{../src/adc.c}
\lstinputlisting{../src/adc.h}

%---------------------------------------------------------------------------
\subsection{UART}
%---------------------------------------------------------------------------

\lstinputlisting{../src/bt_hal.c}
\lstinputlisting{../src/bt_hal.h}
\lstinputlisting{../src/uart.c}
\lstinputlisting{../src/uart.h}
\lstinputlisting{../src/uart_streams.c}
\lstinputlisting{../src/uart_streams.h}

%---------------------------------------------------------------------------
\subsection{Miscellaneous}
%---------------------------------------------------------------------------

\lstinputlisting{../src/common.h}
\lstinputlisting{../src/pong.c}
\lstinputlisting{../src/pong.h}
\lstinputlisting{../src/wiimotes.h}

%---------------------------------------------------------------------------
\subsection{Libs}
%---------------------------------------------------------------------------

\lstinputlisting{../lib/libmp3/mp3.h}
\lstinputlisting{../lib/libmp3/mp3.c}
\lstinputlisting{../lib/libwiimote/uart/util.h}
\lstinputlisting{../lib/libwiimote/uart/hal_wt41_fc_uart.h}
\lstinputlisting{../lib/libwiimote/wii_user/wii_user.c}
\lstinputlisting{../lib/libwiimote/wii_user/wii_user.h}
\lstinputlisting{../lib/libwiimote/wii/wii_bt.h}
\lstinputlisting{../lib/libwiimote/wii/wii_bt.c}
\lstinputlisting{../lib/libwiimote/hci/hci.c}
\lstinputlisting{../lib/libwiimote/hci/hci.h}
\lstinputlisting{../lib/libsdcard/util.h}
\lstinputlisting{../lib/libsdcard/sdcard.h}
\lstinputlisting{../lib/libsdcard/sdcard.c}

%***************************************************************************
}% small
\end{document}
